\subsection{Einfache Anwendungsbeispiele}\label{maschinelleslernen:anwendungsbeispiele}
Es gibt verschiedene Herangehensweisen, meist bietet es sich an, erst mal einen groben �berblick �ber die Daten zu erhalten. F�r die erste Erl�uterung werden Datens�tze angenommen, welche in folgender Datenstruktur vorliegen.

\subsubsection*{Daten aus Datei lesen und anzeigen}
Maschinelles Lernen \randnotiz{Beispieldaten} macht nur Sinn mit entsprechenden Daten. Diese sollten im Besten Fall bereits in einer strukturierten Form vorliegen.
\lstinputlisting[language=Python]{chapters/advancedTopics/src/machinelearning/datasample1.txt}\label{datasample1:lst:datasample}


Nun kann mit der Entwicklung begonnen werden. Um den Datentyp \randnotiz{Datei lesen} mit Daten zu versorgen findet sich im Folgenden ein kleiner Codeausschnitt:
\lstinputlisting{chapters/advancedTopics/src/machinelearning/readdatafromfile.py}\label{readdatafromfile:lst:readdata}

Die aufbereiteten Daten k�nnen im n�chsten Schritt visualisiert werden. Vorher noch eine weitere Variante wie Daten geladen werden k�nnen.

\subsubsection*{Daten aus Paket laden}
Eine\randnotiz{iris-Dataset laden} weitere M�glichkeit ist das Laden von Daten aus Paketen. Hier wird beispielhaft das Lesen aus dem Paket \lstinline$iris$ vorgestellt.
\lstinputlisting[language=Python,firstline=3,lastline=19]{chapters/advancedTopics/src/machinelearning/loadiris.py}\label{loadiris:lst:loadiris}

Nun kann es an die Darstellung der Daten gehen.

\subsubsection*{Plots erzeugen}
Aus den oben gelesen Daten kann nun ein Plot erzeugt werden. F�r das Beispiel oben bietet sich die Darstellung in einem Scatterplot an.


\randnotiz{Plot erzeugen}\lstinputlisting[language=Python]{chapters/advancedTopics/src/machinelearning/defineplot1.py}\label{defineplot1:lst:2dplot}

Eine Darstellung in 3D ist ebenfalls m�glich. Hierzu wird aus dem DataSet noch eine dritte Dimension ausgelesen.

\randnotiz{3 Achsen}\lstinputlisting[language=Python]{chapters/advancedTopics/src/machinelearning/defineplot2.py}\label{defineplot2:lst:3dplot}
