\section{SymPy}
\label{sympy}
SymPy ist eine Python-Bibliothek f�r symbolisch-mathematische Berechnungen.
Sie unterst�tzt laut der Dokumentation unter anderem folgende Funktionen \cite{sympy}:

\begin{itemize}
  \item Einfache symbolische Arithmetik
  \item Differenzialrechnung
  \item Integralrechnung
  \item Algebra
\end{itemize}

Dabei ist zu ber�cksichtigen dass SymPy nicht die Syntax von Python erweitert und ist somit an Limitierungen die in Python vorhanden sind gebunden. Ein Beispiel daf�r ist die implizite Multiplikation wie \lstinline!3x! welche in Python nicht erlaubt ist und daher auch nicht in SymPy.
% Zitat: https://www.sympy.org

Um SymPy zu installieren, kann der Befehl \lstinline!pip install sympy!
verwendet werden.

\subsection{Symbole}
\label{sympy:subsection:symbols}

In SymPy m�ssen Variablen als Symbole definiert werden bevor sie genutzt werden k�nnen:
\footnote{Der Import von \lstinline!sympy! wird der
�bersichtlichkeit halber nachfolgend ausgelassen.}
\begin{lstlisting}
>>> from sympy import *
>>> x = symbols('x')
>>> x + 1
x + 1
\end{lstlisting}

\lstinline!symbols! wird dabei ein Reihe von Symbol Variablennamen �bergeben, die durch Leerzeichen oder Kommata von einender getrennt sind. Diese werden dann Python Variablen zugewiesen. Dabei m�ssen diese Namen nicht �bereinstimmen und k�nnen ebenso l�nger als ein Zeichen lang sein.

\begin{lstlisting}
>>> x, y, z = symbols('x y z')
>>> a, b = symbols('b a')
>>> a
b
>>> square = symbols('triangle')
>>> square + b
triangle + a
\end{lstlisting}

�blicherweise werden Symbole allerdings nach dem Variablennamen benannt um Verwirrungen zu vermeiden.

\subsection{Einfache Operationen}
\label{sympy:subsection:simpleOperations}

In diesem Abschnitt werden einige einfache Operationen behandelt, welche mit SymPy umgesetzt werden k�nnen.

\subsubsection{Gleichheit}
Da SymPy nicht die Syntax von Python erweitert kann das \lstinline!'='! nicht f�r den Gleichheitsvergleich genutzt werden. Ebenso kann \lstinline!'=='! nur bedingt zum Vergleich genutzt werden, da es auf die exakte Strukturelle Gleichheit testet und immer ein \lstinline!bool! als Ergebnis hat und somit keine Gleichung erstellen kann.

\footnote{Symbole wie x werden im Weiteren der �bersichtlichkeit halber als initialisiert angesehen }
\begin{lstlisting}
>>> x + 3 == 4 
False
\end{lstlisting}

Um in SymPy eine Gleichung zu erstellen existiert ein spezielles Objekt \lstinline!Eq!
Eine Gleichung wie \lstinline!x + 3 = 4!kann mit folgender Anweisung erzeugt werden:

\begin{lstlisting}
>>> Eq(x + 3, 4)
Eq(x + 3, 4)
\end{lstlisting}

Um zwei Terme auf mathematische Gleichheit zu pr�fen, ist es einfacher statt \lstinline!a == b! auf \lstinline! a - b = 0! zu testen. Dazu wird \lstinline!a - b! vereinfacht und anschlie�end mit \lstinline!0!verglichen. Dazu wird die Funktion \lstinline!simplify! genutzt, auf diese wird im Verlauf noch genauer eingegangen. Sie vereinfacht den Term soweit wie m�glich.

Beispielsweise mit der Gleichung \(x + 1)^2 = x^2 + 2x + 1\):
\begin{lstlisting}
>>> a = (x + 1)**2
>>> b = x**2 2 * x + 1
>>> simplify(a - b) == 0
True
\end{lstlisting}



\subsubsection{Substitution}
Substitutionen sind einer der Grundlegenden Aktionen die mit einem mathematischen Ausdruck getan werden k�nnen. Bei einer Substitution werden alle Vorkommen von etwas in dem Ausdruck mit etwas neuem ersetzt. Diese Operation wird in SymPy mit der Funktion \lstinline!subs! durchgef�hrt. Beispielsweise wird x durch y ersetzt:

\begin{lstlisting}
>>> expr = x**2 + 2 * x + z
>>> expr.subs(x, y)
y**2 + 2 * y + z
\end{lstlisting}

\warning{Objekte in SymPy sind unver�nderbar. Das bedeutet \lstinline!subs! ver�ndert nicht das bestehende Objekt sonder gibt ein Neues zur�ck!}

Es ist ebenso m�glich mehrere Substitutionen mit einem Aufruf durchzuf�hren dazu wird \lstinline!subs! eine Liste von Paaren �bergeben:

\begin{lstlisting}
>>> expr = x**2 + 2 * x + z
>>> expr.subs([(x, y), (z, 1)])
y**2 + 2 * y + 1
\end{lstlisting}


\subsubsection{String zu SymPy Ausdruck}
Mit der Funktion \lstinline!symify! lasen sich Strings in SymPy Ausdr�cke umwandeln.

\begin{lstlisting}
>>> str_expr = "x ** 2 + 2 * x + 1"
>>> expr = sympify(str_expr)
>>> expr
x**2 + 2 * x + 1
\end{lstlisting}
Dabei ist zu beachten, dass der Ausdruck wohlgeformt ist da zur Berechnung \lstinline!eval! genutzt wird.


\subsection{Vereinfachung}
\label{sympy:subsection:simplification}

In SymPy gibt es verschiedene Methoden um Ausdr�cke zu vereinfachen.

\subsubsection{Vereinfachen}
Die simpelste Art zu vereinfachen ist die Verwendung der \lstinline!simplify!-Methode. Dabei wird versucht die einfachste Darstellung eines Ausdrucks zu finden und dieser wird dann zur�ckgegeben. Beispielsweise folgende Vereinfachungen:

\begin{lstlisting}
>>> simplify(sin(x)**2 + cos(x)**2)
1
>>> simplify((x**3 + x**2 - x - 1)/(x**2 + 2*x + 1))
x - 1
>>> simplify(x**2 + 2*x + 1)
x**2 + 2 * x + 1
\end{lstlisting}

Im letzten Beispiel ist zu sehen dass \lstinline!simplify! nichts ge�ndert hat. M�glicherweise w�re eine gew�nschte Vereinfachung aber in der Form \(x + 1)^2\). Wenn eine bestimmte Form der Vereinfachung gewollt ist es Sinnvoll die zu dieser Form passende Methode zu w�hlen.

\subsubsection{Erweitern}
Des Erweitern eines Ausdrucks in SymPy wird �ber die \lstinline!expand!-Methode realisiert. Untenstehendes Beispiel zeigt wie eine Erweiterung aussehen kann. 

\begin{lstlisting}
>>> expand((x + 1)**2)
x**2 + 2 * x + 1
>>> expand((x + 1)*(x - 2) - (x - 1)*x)
-2
\end{lstlisting}
Wie das Beispiel zeigt kann der Ausdruck durch das Erweitern l�nger oder k�rzer werden. Das h�ngt davon ab in wie weit sich Teile des Ausdrucks gegenseitig aufheben.

\subsubsection{Faktorisieren}
Faktorisiert wird in SymPy mit der \lstinline!factor!-Methode

\begin{lstlisting}
>>> factor(x**2 + 2 * x + 1)
(x + 1)**2
>>> factor(x**2*z + 4*x*y*z + 4*y**2*z)
z*(x + 2*y)**2
\end{lstlisting}

Wie im Beispiel zu sehen ist ist das Faktorisieren das exakte Gegenteil zum Erweitern.
Mit der \lstinline!factor_list!-Methode l�sst sich eine Liste der einzelnen Faktoren ausgeben.

\begin{lstlisting}
>>> factor(x**2*z + 4*x*y*z + 4*y**2*z)
z*(x + 2*y)**2
(1, [(z, 1), (x + 2*y, 2)])
\end{lstlisting}



\subsection{Berechnung}
\label{sympy:subsection:calculus}
Berechnung von Termen

\subsubsection{Differenzieren}


\subsubsection{Integrieren}

\subsubsection{Grenzwerte}



\subsection{Gleichungen L�sen}
\label{sympy:subsection:solvingFunktions}

Funktionen l�sen

\subsection{Matrizen}
\label{sympy:subsection:matrices}

Matrizen

\uebung
\aufgabe{numpy_01}
\aufgabe{numpy_02}
